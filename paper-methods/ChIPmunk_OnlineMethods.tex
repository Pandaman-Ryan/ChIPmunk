\documentclass[12pt]{article}

% Imports
\usepackage{hyperref}
\usepackage[margin=0.5in]{geometry}
\usepackage{ctable}
\usepackage{array}
\usepackage{titlesec}
\usepackage{amsmath}

% Paragraph spacing
\setlength{\parindent}{0em}
\setlength{\parskip}{0.5em}

% Default font
\renewcommand*{\familydefault}{\sfdefault}

% title spacing
\titlespacing*{\section}
{0pt}{2pt}{0pt}
\titlespacing*{\subsection}
{0pt}{2pt}{0pt}

% table lines
\newcolumntype{?}{!{\vrule width 1pt}}

% hyperlinks
\hypersetup{
  breaklinks=true,  % so long urls are correctly broken across lines
  colorlinks=true,
  urlcolor=blue,
  linkcolor=red,
  citecolor=red,
 }

\begin{document}

%%% Notes %%%%
% * Put one sentence per line to make github track changes easier
% * Do not hard code references to figures, equations in this doc. use \ref{label} instead
\section*{Online Methods}

\subsection*{ChIPmunk model}

ChIPmunk models each major step (shearing, pulldown, PCR, and sequencing) of the ChIP-seq protocol (\textbf{Figure 1a, steps 1-4}) as a distinct module. It assumes binding sites for the target epitope and their binding scores (probabilities) are known. Model parameters are summarized in \textbf{Table 1}.

\subsubsection*{1. Shearing}

In step 1, cross-linked DNA is sheared to a target fragment length, for instance by sonication.
ChIPmunk models the length distribution of fragments.
We model fragment lengths using a gamma distribution (\textbf{Figure 1b}) based on empirical observation of fragment distributions which have long right tails (\textbf{Supplementary Figure 2}).
In the case of paired-end reads, fragment lengths can be determined trivially from the mapping locations of paired reads.
For single-end reads, individual fragment lengths are not directly observed.
We outline a novel method below for inferring summary statistics for the length distribution using single-end reads.

\paragraph{Inferring fragment lengths from paired-end reads}
The observed fragment length ($X_i$) for each read pair $i$ can be computed based on the mapping coordinates of the two reads.
The learn module randomly selects 10,000 read pairs from the input BAM for fitting a gamma distribution.
Read pairs are filtered to remove fragments that are unaligned, not properly paired, marked as duplicates or marked as secondary alignments.
Read pairs are further filtered to remove fragments with length greater than 3 times the median length of selected fragments.

The mean fragment length is easily computed as $\mu = \dfrac{\sum_{i=1}^{n}X_i}{n}$, where $n$ is the number of fragments remaining after filtering. We then use the method of moments to find maximum likelihood estimates of the gamma distribution shape ($k$) and scale parameters ($\theta$):

\begin{equation}
  \theta = \dfrac{1}{n\mu}\sum_{i=1}^{n}(X_i - \mu)^2
\end{equation}

\begin{equation}
  k =  \dfrac{\mu}{\theta}
\end{equation}

\paragraph{Inferring fragment lengths from single-end reads}
To estimate the fragment length distribution from single-end reads, we assume the length distribution follows a gamma distribution with mean $\mu$ and variance $v$, and use reads located inside ChIP-seq peaks (provided as input) to estimate $\mu$ and $v$ which are used to compute $k$ and $\theta$.

For each peak $peak_i$, we keep track of two lists, $\{\text{start}\}_{\text{peak}_i}$ and $\{\text{end}\}_{\text{peak}_i}$.
For each read overlapping $\text{peak}_i$, if the read is on the forward strand we add its start coordinate to $\{\text{start}\}_{\text{peak}_i}$.
If the read is on the reverse strand we add its start coordinate to $\{\text{end}\}_{\text{peak}_i}$.
The center point of this peak is calculated as:

\begin{equation} \label{eq:center}
  \text{center}_{\text{peak}_i} = \frac{mean(\{\text{start}\}_{\text{peak}_i}) + mean(\{\text{end}\}_{\text{peak}_i})}{2}
\end{equation}
  
For every $\text{peak}_i$ we offset the coordinates in $\{\text{start}\}_{\text{peak}_i}$ and $\{\text{end}\}_{\text{peak}_i}$ by $\text{center}_{\text{peak}_i}$, so that the coordinates of start points and end points are symmetric around zero.
We then concatenate lists from each peak to form $\{\text{start}\}$ and $\{\text{end}\}$:

\begin{equation} \label{eq:concat}
  \begin{array}{c}
  \{\text{start}\} = \oplus_{i=0}^{n} (\{\text{start}\}_{\text{peak}_i} - \text{center}_{\text{peak}_i})\\
  \{\text{end}\} = \oplus_{i=0}^{n} (\{\text{end}\}_{\text{peak}_i} - \text{center}_{\text{peak}_i})
  \end{array}
\end{equation}

The mean fragment length $\mu$ can be estimated as:
\begin{equation}
  \mu = mean(\{\text{end}\}) - mean(\{\text{start}\})
\end{equation}

We calculate the probability density functions, cumulative density functions and expected density functions for both $\{\text{start}\}$ and $\{\text{end}\}$.
The expected density function $EDF(x)$ is defined as the expected deviation of a random element in the list to $x$:
\begin{equation} \label{eq:EDF}
  \begin{array}{c} 
    EDF_{start}(x) = E(|S - x|) \\
    EDF_{end}(x) = E(|E - x|)
    \end{array}
\end{equation}
where $S$ is a random element in $\{\text{start}\}$ and $E$ is a random element in $\{\text{end}\}$.

After we compute $\mu$, we can reduce the density function of the fragment length distribution from $p_{\mu, v}(x)$ to $p_v(x)$, the probability density function of the fragment length variance. 
We construct a score function $F(v)$ as shown below.
Intuitively, if we have a correct guess of $v$, $F(v)$ should be equal to zero.


\begin{equation}
  \begin{array}{c} \label{eq:Fv}
F(v) = E_v(|S + \frac{L}{2}|) + E_v(|E - \frac{L}{2}|) - E(|S + \frac{\mu}{2}|) - E(|E- \frac{\mu}{2}|) \\
E_v(|S + L/2|) = \sum_{x=0}^\infty p_v(x) * EDF_{\text{start}}(-\frac{x}{2}) \\
E_v(|E - L/2|) = \sum_{x=0}^\infty p_v(x) * EDF_{\text{end}}(\frac{x}{2}) \\
E(|S + \frac{\mu}{2}|)=EDF_{\text{start}}(x) \\
E(|E - \frac{\mu}{2}|)=EDF_{\text{end}}(x) \\
\end{array}
\end{equation}

where $L$ represents a randomly chosen fragment length.

To find an optimal $v$ that minimizes $|F(v)|$, we conduct a binary search between 1000 and 10,000.

In practice, we slightly offset the last two items in the score function in \textbf{Equation~\ref{eq:Fv}} to get the score below, which gives slightly more accurate estimation of $v$ on real data.
This may be due to the fact that fragment length distributions are truncated on the left end, with little or no fragments with lengths less than 100bp observed, and thus do not follow a true gamma distribution.

\begin{equation}
F(v) = E_v \left( \left| S + \frac{L}{2} \right| \right) + E_v \left( \left| E - \frac{L}{2} \right| \right) - E\left( \left| S + \frac{\mu}{2} - \frac{E- \frac{\mu}{2}}{4} \right| \right) - E \left(\left| E- \frac{\mu}{2} - \frac{S + \frac{\mu}{2}}{4} \right|\right)
\end{equation}

\subsubsection*{Pulldown}
In step 2, sheared cross-linked DNA is subject to pulldown, during which an antibody for the protein or modification of interest is used to enrich the pool of fragments for those bound to the epitope recognized by the antibody.
This process is imperfect: some bound fragments will not be pulled down, and some unbound fragments will be pulled down.
To model this process, we quantify the ratio of the probability of pulling down a bound vs. unbound fragment as described below.
This value is specific to each ChIP-seq experiment and depends on the antibody specificity as well as the fraction of the genome bound by the factor of interest.

We use $B_i$ to denote that fragment $i$ is bound, $\overline{B_i}$ to denote that fragment $i$ is unbound, and $D_i$ to denote that fragment $i$ is pulled down. For fragment $i$, assuming the probability that a given bound fragment is pulled down is approximately constant over all fragments, the probability of being pulled down can then be written as:
\begin{equation} \label{eq:pulldown}
  P(D_i) = P(D, B_i) + P(D, \overline{B_i}) = P(D|B)P(B_i) + P(D|\overline{B})(1-P(B_i))
\end{equation}

where $P(D|B)$ denotes the probability that a given bound fragment will be pulled down and $P(D|\overline{B})$ denotes the probability that a given unbound fragment will be pulled down.

For any fragment $i$, we set the probability of it being bound $P(B_i)$ based on the scores of peaks it overlaps:

\begin{equation}
  P(B_i) = \begin{cases}
    0, & \text{ if the fragment doesn't overlap any peaks} \\
    1-\Pi_{r \in R} (1-S(r)), & \text {if the fragment overlaps one or more peaks}
    \end{cases}
\end{equation}

where $R$ is the set of all peak regions overlapping fragment $i$ and $S(r)$ is the probability that peak $r$ is bound. A method for estimating $S(r)$ is detailed in the ``ChIPmunk implementation'' section below.

We compute conditional pulldown probabilities using Bayes' Rule:
\begin{equation}
  P(D|B) = \frac{P(B|D)P(D)}{P(B)}
\end{equation}
\begin{equation}
  P(D|\overline{B}) = \frac{P(\overline{B}|D)P(D)}{P(\overline{B})}
\end{equation}

where here $P(B|D)$ and $P(\overline{B}|D)$ represent averages across all fragments.
We do not have a straightforward way to compute $P(D)$, the average probability that a fragment is pulled down, using only observed ChIP-seq reads since we do not actually observe fragments that are not pulled down.
Thus, we do not attempt to compute these conditional probabilities directly.
Instead, we take the ratio which cancels $P(D)$:

\begin{equation}
\alpha = \frac{P(D|B)}{P(D|\overline{B})} = \frac{P(B|D)P(\overline{B})}{P(\overline{B}|D)P(B)}
\end{equation}

% Define f
$P(B)$, or the probability that a fragment is bound on average, is equal to $f$, the fraction of the genome bound by the factor of interest.
We can approximate $f$ as the sum of the lengths of all peaks $l_k$ weighted by their binding probabilities divided by the total length of the genome $G$:

\begin{equation}
  f \approx \frac{1}{G} \sum_{r \in R} S(r)l_r
\end{equation}

where $G$ is the size of the reference genome, $S(r)$ is the probability peak region $r$ is bound as described above, and $l_r$ is the length of peak $r$, assuming no overlap between peaks. 

% Define s
$P(B|D)$, or the probability that a fragment is bound given that it is pulled down, is a measure of the specificity of the antibody. Assuming the majority of reads falling in peaks are truly bound, we can roughly approximate this as the percent of fragments falling within peaks, which we denote as $s$.

% Ratio
Using these two metrics, $f$, and $s$ (\textbf{Supplementary Table 1}), we can simplify the ratio $\alpha$ as:
\begin{equation} \label{eq:ratiosimple}
  \alpha = \frac{s(1-f)}{(1-s)f}
\end{equation}

% How it's done in practice
Since we only know the ratio $\alpha$, for simplicity in simulations we set $P(D|B)=1$ and $P(D|\overline{B})=\frac{1}{\alpha}$.
Substituting into Equation~\ref{eq:pulldown} above, we compute the probability that fragment $i$ is pulled down as:

\begin{equation}
  P(D_i) = P(B_i) + \frac{1}{\alpha}(1-P(B_i))
\end{equation}
where $P(B_i)$ is based on peak scores as defined above.

Note, in reality, $P(D|B)$ is likely to be much less than 1, since the pulldown process is inefficient and many fragments are lost.
Further, the number is likely to vary largely across different experiments.
Estimating the absolute value of $P(D|B)$ from real datasets is a topic of future work.

\subsubsection*{PCR}

In step 3, PCR is used to amplify pulled down fragments before sequencing.
Let $n_i$ represent the number of reads (or read pairs) with $i$ PCR duplicates (including the original fragment).
$n_i$ is modeled using a geometric distribution, where $p$ gives the probability that a fragment has no PCR duplicates.
The parameter $p$ is estimated as $1/\overline{n}$, where $\overline{n} = \frac{\sum_{i=1}^\infty (i* n_i)}{\sum_{i=1}^\infty n_i}$.

\subsubsection*{Sequencing}

In step 4, amplified fragments are subject to sequencing.
Sequences are based on an input reference genome using the coordinates of each fragment.
We model the per-base pair substitution rate, insertion rate, and deletion rate (\textbf{Supplementary Table 1}).

\subsection*{ChIPmunk implementation}

\subsubsection*{Peak scores}
A peak score ($S(r)$) is defined for each input peak, where $S(r)$ gives the probability that a fragment overlapping the region is bound.
Note we cannot directly estimate this probability from bulk ChIP-seq data, but assume variability in intensity across peaks is representative of different relative binding probabilities.
Based on user input, ChIPmunk computes these binding probabilities either based on peak intensities given in an input BED file or based on read counts from an existing BAM file.
If peak intensities are provided, $S(r)$ is computed as the score of peak $r$ divided by the maximum peak intensity.
If a BAM file is provided, $S(r)$ is defined as the number of reads overlapping peak $r$ divided by the maximum number of reads overlapping any peak.
In both cases, resulting scores $S(r)$ are between 0 and 1.

By specifiying the option -{}-no-scale, users may directly input binding scores that will be treated directly as probabilities and will not be rescaled.
Users may also specify -{}-scale-outliers to remove peak intensities greater than 3 times the median score before rescaling peak intensities to reduce the effect of outlier peaks. In this case all peaks with scores greater than 3 times the median will be set to 1.

\subsubsection*{Learn implementation}
The ChIPmunk learn module takes a set of input peaks (BED) and aligned reads (BAM) and learns parameters for (1) fragment length distribution ($k$, $\theta$), (2) pull-down efficiency ($f$, $s$), and (3) PCR efficiency ($p$) (\textbf{Supplementary Table 1}).
BAM files must have PCR duplicates marked using a tool such as Picard \cite{picard} to enable accurate estimation of PCR parameters.
The learn module  outputs model parameters to a JSON file that can be used as input to the simreads module.

\subsubsection*{Simulation implementation}

The simreads module takes in a set of peaks, model parameters (\emph{e.g.}, from the learn module), and experimental parameters (including read length, number of reads) and simulates raw sequencing reads in FASTQ format.
The user must additionally specify the number of simulation rounds, which denotes the number of times the input reference genome is processed by ChIPmunk. Notably, this number is related to but not directly comparable to the number of experimentally processed cells, since pulldown efficiency is not directly included in our current model
We have found that values of 25-100 simulation rounds work well in most settings for HMs, and 1,000 rounds works well for TFs.

First, each copy of the genome (equivalent to one simulation round) is randomly fragmented based on the specified fragment length gamma distribution.
Next, we decide whether to pull down each fragment based on its overlap with input peaks based on $P(B_i)$ described above.
Then, pulled down fragments are subject to PCR based on the specified PCR rate.
Finally, we generate reads from the resulting pool of fragments based on input parameters (using the specified mode read length, number of reads, and mode [single/paired]).

In practice, in each round the genome is processed in bins (default size 100kb) to avoid storing large fragment pools in memory.
In a preprocessing step, we determine how many reads to generate from each simulation round based on the target number of output reads.
ChIPmunk is parallelized by performing different simulation rounds on separate threads simultaneously.

\subsubsection*{C++ implementation details}
ChIPmunk is implemented as an open source C++ project with source code publicly available on Github: https://github.com/gymreklab/ChIPmunk.
It uses the open source libraries htslib (www.htslib.org) for parsing BAM files and POSIX threads for multithreading.
It supports standard file formats including BAM/CRAM for aligned reads, BED for peak files, JSON for model files, and FASTQ for raw reads.
All analyses were performed using ChIPmunk v1.9.

\subsection*{Inferring parameters of ENCODE datasets}

We used the ENCODE Project's REST API to write a Python script to automatically identify ChIP-sequencing datasets available for the GM12878 cell line.
For each replicate of each experiment, we fetched unfiltered BAM files and narrowPeak or broadPeak peak files for transcription factors and histone modifications, respectively (output type=''peaks'' or ``peaks and background as input for IDR".
We additionally fetched narrowPeak/broadPeak files based on combined replicate information (output type = ``optimal idr thresholded peaks'') for use as ground truth peak sets for power analyses.

We used the Picard \cite{picard} MarkDuplicates tool (v2.18.11) with default parameters to mark duplicates in each BAM file.
For paired end datasets, we called ChIPmunk's learn module with non-default parameter -{}-paired.
For single end datasets, we used non-default parameters -c 7 -{}-thres 100 for transcription factors and -c 7 -{}-thres 5 for histone modifications.
For 6 example paired-end datasets, we called learn in both paired end and single end mode to compare inferred fragment length distributions (\textbf{Supplementary Figure 2}).

Inferred parameters as well as links to JSON model files that can be used as input to the ChIPmunk simreads module are provided in \textbf{Supplementary Table 2}.

\subsection*{Evaluating ChIPmunk using ENCODE datasets}

We performed simulations using an increasing number of simulation rounds (ChIPmunk argument -{}-numcopies 1, 5, 10, 25, 50, 100, or 1000) for two factors (the histone modification H3K27ac and the transcription factor CTCF) with ENCODE data available in the GM12878 cell line. (BAM accession ENCFF097SQI and broadPeak [BED] accession ENCFF465WTH for H3K27ac; BAM accession ENCFF598OOE and narrowPeak [BED] accession ENCFF706QLS for CTCF.
For each number of simulations in each factor, we ran ChIPmunk's simreads tool on hg19 chromosome 19 (-{}-region chr19:1-59128983) using the corresponding learned ENCODE model (see \textbf{Supplementary Table 2}) and the same sequencing parameters as the ENCODE reads from chromosome 19 for each dataset (-{}-numreads 269113 -{}-readlen 51 for H3Ka7ac; -{}-numreads 521557 -{}-readlen 36 for CTCF). 
We additionally used options -p ENCODE.bed -t bed -c 7, where ENCODE.bed represents the corresponding peak file for each factor with accessions listed above.
Simulated reads were aligned to the hg19 reference genome using BWA MEM \cite{bwamem} v0.7.12-r1039 with default parameters and converted to sorted and indexed BAM files using samtools \cite{samtools} version 1.5. Duplicates were flagged using the Picard \cite{picard} MarkDuplicates tool (v2.18.11) with default parameters.

We used the bedtools \cite{bedtools} (v2.27.1) makewindows command to generate a list of non-overlapping windows cross chromosome 19 and the bedtools multicov command to count the number of reads from each BAM file falling in each window. We used bins of size 5kb to evaluate H3K27ac and 1kb to evaluate CTCF, since HMs tend to have much broader peaks than TFs.
We used the bedtools intersect command to determine the intersection of each bin with the input peak files. We determined the Pearson correlation between log10 read counts in each bin for the simulated vs. ENCODE dataset after removing bins with 0 counts in each dataset and adding a pseudocount of 1 read to each bin.

Timing experiments were performed in a Linux environment running Centos 7.4.1708 on a server with 28 cores (Intel\textsuperscript{\textregistered} Xeon\textsuperscript{\textregistered} CPU E5-2660 v4 @ 2.00GHz) and 125 GB RAM using the UNIX ``time'' command and are based on the ``sys'' time reported. 

\subsection*{Comparison to ChIPulate}

We compared performance ChIPulate \cite{chipulate} vs. ChIPmunk on CTCF (bam=ENCFF406XWF bed=ENCFF833FTF) and H3K27ac (bam=ENCFF385RWJ bed=ENCFF816AHV).

For ChIPulate, we first computed binding energies for each peak required as input. Binding probabilities $S(r)$ for each peak $r$ were computed as described above by scaling read counts from the ENCODE BAM files in each peak to be between 0 and 1. Then we computed binding energies for each region $r$ as

$$E^{\text{bound}}_r= E^{\text{unbound}}_r + \text{chemical potential} + \ln\left[(1-S(r))/S(r)\right]$$

$E^{\text{unbound}}_i$ and chemical potential are provided as command line inputs to ChIPulate. We set chemical potential to 0, which provided the best dynamic range across peak scores and highest correlation with real data. Background binding energy was set to the default (1.0).

ChIPulate parameters were set to achieve similar fragment length distributions, PCR rates, number of reads, and read length used for the ENCODE data and ChIPmunk simulations. We used the following settings for running ChIPulate to simulate CTCF based on an input set of peaks identified by ENCODE (accession ENCFF833FTF): p\_amp=0.93, p\_ext=0.54, -{}-mu-A=0, -d=332, -{}-fragment-length=182, -{}-fragment-jitter=39, -{}-read-length=36. We used the following settings for running ChIPulate to simulate H3K27ac based on an input set of peaks identified by ENCODE (accession ENCFF816AHV): p\_amp=0.95, p\_ext=0.54, -{}-mu-A=0, -d=101, -{}-fragment-length=211, -{}-fragment-jitter=90, -{}-read-length=51.
In both cases we used 1 million cells (-n 1000000) but found results and run time did not vary noticeably when using fewer cells.

To run ChIPmunk on the same dataset, we first learned parameters using the ENCODE BAM and peak files, then ran simreads using the learned model and peak files as input, with parameters: -{}-numcopies 1000 -{}-numreads 521557 -{}-readlen=36 for CTCF and -{}-numcopies 25 -{}-numreads 269113 -{}-readlen=51 for H3K27ac.

Read counts in windows of 1kb and 5kb for CTCF and H3K27ac respectively were compared between simulated and ENCODE datasets using the bedtools multicov command as described above.

\subsection*{Power analysis}
We used ChIPmunk's simreads tool to generate genome-wide ChIP-seq datasets for five histone modifications (HMs) (H3K27ac, H3K4me3, H3K4me1, H3K27me3, and H3K36me3) and two transcription facetors (TFs) (BACH1 and CTCF) based on model parameters ($f$ and $s$) learned from available ENCODE datasets

BAM/BED accssions ENCFF385RWJ/ENCFF816AHV, ENCFF398NET/ENCFF795URC, \\ENCFF809FFP/ENCFF851UKZ, ENCFF677IFV/ENCFF921LKB, ENCFF191SDM/ENCFF479XLN,\\ ENCFF518TTP/ENCFF012JXJ, and ENCFF406XWF/ENCFF833FTF for
H3K27ac, H3K4me3, H3K27me3, H3K4me1, H3K36me3, BACH1, and CTCF, respectively. BED accessions are for broadPeak files for the HMs and narrowPeak files for the TFs.

For each HM and TF, we used options -p ENCODE.bed -t bed -c 7 where ENCODE.bed represents the corresponding peak file for each factor with accessions listed above and set -{}-gamma-frag 10,20 -{}-pcr\_rate 0.85 -{}-numcopies 100 -{}-readlen 100 to keep these parameters constant across all runs.
We simulated datasets using numbers of reads (-{}-numreads) set to 1, 5, 10, 50, or 100 million reads.
We additionally simulated a control dataset (whole cell extract; WCE) with the same number of reads but using the options -t wce and no input BED file. As described above, simulated reads were aligned to hg19 using BWA-MEM and duplicates were flagged with Picard.
We used MACS2 \cite{MACS2} to call peaks on the resulting duplicate-flagged files using whole cell extract reads as a control and with default options except -{}-broad for histone modifications.

We used the bedtools \cite{bedtools} (v2.27.1) intersect command to determine the percentage of input ENCODE peaks overlapping a peak called by MACS2 from the simulated data, using narrowPeak output for transcription factors and broadPeak output for histone modifications.

\subsection*{Analysis of spike in controls}
We used the bedtools \cite{bedtools} (v2.27.1) random command to generate 10,000 random peaks of size 100 from hg19 chromosome 19 and 20 random peaks of size 100 from Drosophila (dm6) chromosome 4. All peaks were initially given a binding score of 1 (meaning 100\% probability of binding).

We then used the ChIPmunk uitlity script chipmunk-spike-in.sh to generate a mock reference genome consisting of the concatenated dm6 and hg19 references and a peak file consisting of both the hg19 and dm6 peaks. We then generated peak files representing different conditions with overall reduced binding of the human sites by setting the scores of all peaks to 0.01, 0.1, 0.25, 0.5, or 0.8, representing reduced binding affinity.
For each condition, we used ChIPmunk's simreads tool with options -p concatenated\_peakds.bed -t bed -c 4 -f mock\_genome.fa -{}-numcopies 100 -{}-noscale -{}-recomputeF -{}-numreads 1000000 and a model file with f: 15, theta: 20, pcr\_rate: 1, and s: 0.4. Note the option -{}-recomputeF is a convenience option to recompute the value of $f$ based on the user input set of peaks.

Reads were aligned to the mock reference genome using BWA MEM. Read densities around peaks were visualized using the annotatePeaks.pl script available from HOMER \cite{HOMER} using options -size 1000 -hist 1.
Because exactly 1 million reads were simulated in each case, mean read counts per bin represent reads per million (RPM).
To compute reference corrected reads per million (RRPM) based on Orlando, \emph{et al}. \cite{spikein}, we divided each value by the number of reads from the condition mapping to the dm6 genome.

\bibliography{ChIPmunk_OnlineMethods}
\bibliographystyle{ieeetr}

\end{document}

